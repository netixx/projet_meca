\documentclass[a4paper,11pt]{article}
\usepackage[utf8]{inputenc}
\usepackage[francais]{babel}
\usepackage[T1]{fontenc}

\usepackage[toc,page]{appendix}
%fonts pour les math
\usepackage{amsfonts}
\usepackage{amsmath}
\usepackage{dsfont}
\usepackage{stmaryrd}
%couleur dans le doc
\usepackage[dvipsnames]{xcolor}
\usepackage{ulem}
%\usepackage{lastpage}
%pour les images
	%\usepackage[dvips, pdftex]{graphicx}
	%\usepackage[section]{placeins}
	%\usepackage{here}
	%\usepackage{float}

%définition du titre du document
\newcommand{\titleinfo}{\textcolor{MidnightBlue}{MEC431 – Projet}}
 %\newcommand{\sectionprefix}{Q}

\usepackage{tikz}
\usepackage{url}
\usepackage{hyperref}
%\usepackage{pst-plot}
%\usepackage{pst-grad}
%\usepackage{pst-node}
\usepackage{booktabs}
%formattage de la page
	\usepackage{geometry}
	\geometry{top=2cm, bottom=2cm, left=2cm, right=2cm}
	%fomattage de l'entête et du pied de page
	\usepackage{fancyhdr}
		\setlength{\headheight}{15.2pt}
		\lhead{\titleinfo}
		\rhead{\leftmark}
	\pagestyle{fancy}


%formattage du titre des sections
	%\usepackage{titlesec}
	%\titleformat{\section}[runin]{\normalfont\bfseries}{\sectionprefix \thesection}{1pt}{}[.]

%longueur des sauts de paragraphe
\setlength{\parskip}{1ex}

%commande pour les ensembles
\newcommand{\R}{\mathbb{R}}
\newcommand{\N}{\mathbb{N}}
\newcommand{\C}{\mathbb{C}}

%commande pour les notations de map
	%\newcommand{\I}{\mathds{1}}
	%\newcommand{\p}{\mathds{P}}
	%\newcommand{\E}{\mathds{E}}
	%\renewcommand{\P}{\mathds{P}}


%commande pour les opérateurs
\newcommand{\INT}{\displaystyle\int}
\newcommand{\SUM}{\displaystyle\sum}
\newcommand{\FRAC}{\displaystyle\frac}
\newcommand{\PROD}{\displaystyle\prod}
\newcommand{\INF}{\displaystyle\inf}

\newcommand{\tens}{\uuline}
\newcommand{\verseur}[1]{\uline{e_{#1}}}
\newcommand{\diage}[1]{\uline{e_{#1}} \otimes \uline{e_{#1}}}
\newcommand{\diveleme}[3]{\FRAC{\partial #1_{#2#3}}{\partial #3} e_{#2}}
\newcommand{\dive}{\mathop{\rm div}}
\newcommand{\grad}{\mathop{\rm grad}}
\begin{document}

\title{\titleinfo}
\author{
	François \bsc{Espinet}
	\\
	Thomas \bsc{Ferreira de lima}
	\\
	Sophie \bsc{Rousselle}
	\\
	Oscar \bsc{Flores Altamirano}
}
\date{}
\maketitle

\section{Calcul de structure}



\subsection{}

$\beta$ est liée a $\delta$ de la façon suivante

\begin{center}
\begin{array}{rll}
\begin{displaymath}
& \beta^2 & = \FRAC{1}{\mu}+\frac{1}{\left(1+\delta\right)^2}\left(\lambda^2-\frac{1}{\mu}\right)  \\
&  & \displaystyle \approx  \frac{1}{\mu}+\left(1-2\delta\right)\left(\lambda^2-\frac{1}{\mu}\right) \\
\Rightarrow & \beta & \displaystyle \approx \left[\frac{1}{\mu} + \left(1-2\delta \right)\left(\lambda^2 -\frac{1}{\mu} \right) \right]^{\frac{1}{2}} \\
& &\displaystyle \approx \lambda\left[1+\left(\frac{1}{\lambda^2\mu}-1 \right)\delta \right]\\
\end{displaymath}
\end{array}

\end{center}

Comme $\delta<<1$ on peut prendre $\beta=\lambda$ et on trouve

$$\underline{\underline{F}} = \frac{1}{\lambda\mu}\underline{e}_r\otimes\underline{e}_R+\lambda\underline{e}_\theta\otimes\underline{e}_\Theta+\mu\underline{e}_z\otimes\underline{e}_Z $$
	
\subsection{}
On fait le calcul directement: $$I_1=\text{tr}\left(\underline{\underline{B}}\right) = \text{tr}\left(\underline{\underline{F}}\cdot{}^{t}\underline{\underline{F}}\right) = \text{tr}\left(\frac{1}{\lambda^2\mu^2}\underline{e}_R\otimes\underline{e}_R + \lambda^2\underline{e}_\Theta\otimes\underline{e}_\Theta + \mu^2\underline{e}_Z\otimes\underline{e}_Z \right)=\left(\frac{1}{\lambda\mu}\right)^2+\lambda^2+\mu^2$$
	
	$$I_2 = \frac{1}{2}\left(I_1^2+\text{tr}\left(\underline{\underline{B}}\cdot\underline{\underline{B}}\right)\right) = \frac{1}{2}\left(I_1^2+\text{tr}\left( \frac{1}{\lambda^4\mu^4}\underline{e}_R\otimes\underline{e}_R + \lambda^4\underline{e}_\Theta\otimes\underline{e}_\Theta + \mu^4\underline{e}_Z\otimes\underline{e}_Z \right)\right)  = \frac{1}{\mu^2}+\frac{1}{\lambda^2}+\lambda^2\mu^2$$
	
\end{enumerate}

%appendice éventuel
%\clearpage
%\rhead{Annexe}
%\renewcommand{\appendixpagename}{\centering{Annexe : Liste des fonctions Scilab}}
%\renewcommand{\labelitemi}{$\bullet$}
%\renewcommand{\labelitemiii}{$\cdot$}
%\renewcommand{\labelitemii}{$\diamond$}
%\renewcommand{\labelitemiv}{$\ast$}
%\begin{appendices}
%\end{appendices}
\end{document}
