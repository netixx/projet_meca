\documentclass[a4paper,11pt]{article}
\usepackage[utf8]{inputenc}
\usepackage[francais]{babel}
\usepackage[T1]{fontenc}

\usepackage[toc,page]{appendix}
%fonts pour les math
\usepackage{amsfonts}
\usepackage{amsmath}
\usepackage{dsfont}
\usepackage{stmaryrd}
%couleur dans le doc
\usepackage[dvipsnames]{xcolor}
\usepackage{ulem}
%\usepackage{lastpage}
%pour les images
	%\usepackage[dvips, pdftex]{graphicx}
	%\usepackage[section]{placeins}
	%\usepackage{here}
	%\usepackage{float}

%définition du titre du document
\newcommand{\titleinfo}{\textcolor{MidnightBlue}{MEC431 – Projet}}
 %\newcommand{\sectionprefix}{Q}

\usepackage{tikz}
\usepackage{url}
\usepackage{hyperref}
%\usepackage{pst-plot}
%\usepackage{pst-grad}
%\usepackage{pst-node}
\usepackage{booktabs}
%formattage de la page
	\usepackage{geometry}
	\geometry{top=2cm, bottom=2cm, left=2cm, right=2cm}
	%fomattage de l'entête et du pied de page
	\usepackage{fancyhdr}
		\setlength{\headheight}{15.2pt}
		\lhead{\titleinfo}
		\rhead{\leftmark}
	\pagestyle{fancy}


%formattage du titre des sections
	%\usepackage{titlesec}
	%\titleformat{\section}[runin]{\normalfont\bfseries}{\sectionprefix \thesection}{1pt}{}[.]

%longueur des sauts de paragraphe
\setlength{\parskip}{1ex}

%commande pour les ensembles
\newcommand{\R}{\mathbb{R}}
\newcommand{\N}{\mathbb{N}}
\newcommand{\C}{\mathbb{C}}

%commande pour les notations de map
	%\newcommand{\I}{\mathds{1}}
	%\newcommand{\p}{\mathds{P}}
	%\newcommand{\E}{\mathds{E}}
	%\renewcommand{\P}{\mathds{P}}


%commande pour les opérateurs
\newcommand{\INT}{\displaystyle\int}
\newcommand{\SUM}{\displaystyle\sum}
\newcommand{\FRAC}{\displaystyle\frac}
\newcommand{\PROD}{\displaystyle\prod}
\newcommand{\INF}{\displaystyle\inf}

\newcommand{\tens}{\uuline}
\newcommand{\verseur}[1]{\uline{e_{#1}}}
\newcommand{\diage}[1]{\uline{e_{#1}} \otimes \uline{e_{#1}}}
\newcommand{\diveleme}[3]{\FRAC{\partial #1_{#2#3}}{\partial #3} e_{#2}}
\newcommand{\dive}{\mathop{\rm div}}
\newcommand{\grad}{\mathop{\rm grad}}
\begin{document}

\title{\titleinfo}
\author{
	François \bsc{Espinet}
	\\
	Thomas \bsc{Ferreira de lima}
	\\
	Sophie \bsc{Rousselle}
	\\
	Oscar \bsc{Flores Altamirano}
}
\date{}
\maketitle

\section{Calcul de structure}



\subsection{}

$\beta$ est liée a $\delta$ de la façon suivante

\begin{center}
\begin{array}{rll}
\begin{displaymath}
& \beta^2 & = \FRAC{1}{\mu}+\frac{1}{\left(1+\delta\right)^2}\left(\lambda^2-\frac{1}{\mu}\right)  \\
&  & \displaystyle \approx  \frac{1}{\mu}+\left(1-2\delta\right)\left(\lambda^2-\frac{1}{\mu}\right) \\
\Rightarrow & \beta & \displaystyle \approx \left[\frac{1}{\mu} + \left(1-2\delta \right)\left(\lambda^2 -\frac{1}{\mu} \right) \right]^{\frac{1}{2}} \\
& &\displaystyle \approx \lambda\left[1+\left(\frac{1}{\lambda^2\mu}-1 \right)\delta \right]\\
\end{displaymath}
\end{array}

\end{center}

Comme $\delta<<1$ on peut prendre $\beta=\lambda$ et on trouve

$$\underline{\underline{F}} = \frac{1}{\lambda\mu}\underline{e}_r\otimes\underline{e}_R+\lambda\underline{e}_\theta\otimes\underline{e}_\Theta+\mu\underline{e}_z\otimes\underline{e}_Z $$
	
\subsection{}
On fait le calcul directement: $$I_1=\text{tr}\left(\underline{\underline{B}}\right) = \text{tr}\left(\underline{\underline{F}}\cdot{}^{t}\underline{\underline{F}}\right) = \text{tr}\left(\frac{1}{\lambda^2\mu^2}\underline{e}_R\otimes\underline{e}_R + \lambda^2\underline{e}_\Theta\otimes\underline{e}_\Theta + \mu^2\underline{e}_Z\otimes\underline{e}_Z \right)=\left(\frac{1}{\lambda\mu}\right)^2+\lambda^2+\mu^2$$
	
	$$I_2 = \frac{1}{2}\left(I_1^2+\text{tr}\left(\underline{\underline{B}}\cdot\underline{\underline{B}}\right)\right) = \frac{1}{2}\left(I_1^2+\text{tr}\left( \frac{1}{\lambda^4\mu^4}\underline{e}_R\otimes\underline{e}_R + \lambda^4\underline{e}_\Theta\otimes\underline{e}_\Theta + \mu^4\underline{e}_Z\otimes\underline{e}_Z \right)\right)  = \frac{1}{\mu^2}+\frac{1}{\lambda^2}+\lambda^2\mu^2$$


\subsection{}
\begin{enumerate}
\item[a]
La courbe $\mu(\lambda)$ nous indique que l'allongement de la partie gonflée du ballon est à peu près proportionnel au grossissement du diamètre (pente affine). La partie constante que l'on trouve vers $\lambda = 1$ indique que le ballon à une tendance à se gonfler par son rayon plutôt que par le gonflement d'un longueur plus grande de ballon, du moins dans la phase initiale du gonflage.

\hspace{0.8cm} Étudier le gonflement du ballon revient donc à étudier l'élargissement d'une section du ballon.

\hspace{0.8cm} On retrouve les deux premières étapes décrites dans l'introduction pour toutes les lois : d'abord la pente de $\Delta p (V/V_{ini})$ est forte, c'est-à-dire qu'il faut appliquer un fort différentiel de pression pour obtenir un faible accroissement du volume du ballon (un faible gonflage donc). Ensuite, la pente devient négative, il faut donc appliquer de moins en moins de différentiel de pression pour obtenir un gonflement supérieur.

\hspace{0.8cm}Enfin, les lois de Gent et la loi de Langevin confirment l'observation faite, il devient plus dur de gonfler le ballon une fois qu'on atteint un $\lambda$ important. Il parait logique d'avoir une divergence du différentiel de pression vu qu'un gonflement infini du ballon n'est pas possible (explosion), la loi néo-hookienne ne prévoit pas ce comportement.

On retrouve ces observations sur $\Delta p (\lambda)$.

\item[b]
En pratique, on observe bien ce comportement lors du gonflement du ballon.

\item[c]
La loi néo-hookienne parait impossible à admettre pour des gonflements importants puisqu'elle ne prévoit pas la divergence des fonctions lorsque $\lambda$ devient grand. Elle reste cependant valable pour de très petits gonflements.

\hspace{0.8cm}Les deux autres lois sont quasi-identiques en comportement, elles diffèrent seulement dans les valeurs, les valeurs de la loi de Gent étant supérieures à celle de la loi de Langevin.

\item[d]
L'intérêt de tirer sur le ballon permet de changer le comportement décrit au début de l'item a. En effet, lorsqu'on tire sur le bout du ballon, il acquiert une propension à augmenter la zone "gonflée" ($\mu$ plus important) plutôt que d'augmenter son rayon : il absorbe l'air ajouté en augmentant la zone gonflée plutôt qu'en augmentant le rayon de la zone déjà gonflée.

\hspace{0.8cm}Cette action à pour effet de garder $\lambda$ proche de 1 lors du gonflement puisque l'air ajouté est réparti en longueur dans le ballon (par le biais de $\mu$), le différentiel de pression à appliquer reste donc faible et le différentiel de pression reste petit, le ballon est donc plus facile à gonfler. On change ainsi le comportement "naturel" du ballon en tirant sur ces fibres de façon orthogonale à la force qu'applique le différentiel de pression sur le ballon, et ce nouveau comportement rend le gonflement plus facile.
\end{enumerate}
%appendice éventuel
%\clearpage
%\rhead{Annexe}
%\renewcommand{\appendixpagename}{\centering{Annexe : Liste des fonctions Scilab}}
%\renewcommand{\labelitemi}{$\bullet$}
%\renewcommand{\labelitemiii}{$\cdot$}
%\renewcommand{\labelitemii}{$\diamond$}
%\renewcommand{\labelitemiv}{$\ast$}
%\begin{appendices}
%\end{appendices}
\end{document}
